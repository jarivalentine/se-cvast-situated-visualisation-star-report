% introduction to the general topic
% background and motivation
% description of the addressed scientific problem
% explanation of fundamental terms and basic definitions
% outline of the manuscript

\section{Introduction}

Situated visualization (SV) is a data representation that present data visually, typically on a screen, and in proximity to its physical data referent in the real worlds.
The concept is relevant across several fields, including Human-Computer
Interaction (HCI) and Information Visualization, and is being extensively explored in areas like augmented reality (AR) and ubiquitous computing.
For instance, an urban planner might use a system like SiteLens to overlays real-time data, such as carbon monoxide (CO) readings, onto a street to identify higher CO levels correspond to idling cars.

The development of situated visualization is part of the ongoing, strong trend toward augmented and mixed reality technologies, which are rapidly improving in quality and expanding beyond entertainment to realize more purposeful applications.
While this trend is beneficial, it has led to a fragmented understanding of terminology in the field.
Without a clear overview of existing approaches, designers must individually research the field, repeatedly encountering different problems that could be avoided if the findings were systematically documented.

\subsection{Problem}

The central problem in this field, as highlighted in research literature, is the inconsistent use of terminology.
This wide appropriation of the concept has resulted in a "disconnected terminology" and leaves interpretations of what situated visualization unclear in current literature.
Willet et al. (2017) also noted that the language necessary to describe and compare different approaches remains limited.
This inconsistency is further complicated by the historical evolution of related concept, such as "location-based services", which were prominent as early as 2004.
As terminology shifts, previous exploration of similar concepts can easily be neglected, forcing researchers to "repeat challenges".
As Bawden (2001) argued, any terminology built on this concept is in danger of being constructed on "shifting sands".

Beyond the terminology issues, several knowledge gaps remain that require systematic exploration, including the categorization of approaches, the relationships between predecessors and successors, and, importantly, the ethical considerations. \newline

For clarity, this paper is grounded in the foundational definitions of the field.
The broadest definition, by White and Feiner (2009), states that situated visualization ``is related to and displayed in its environment".
This was built upon by Willet et al. (2017), who defined it through the introduction of the ``physical data referent".
We note that related terms found in practice include, but are not limited to: on-site, in-situ, ambient, ubiquitous, location-based, embodied, embedded, and contextual visualization.
The focus of this paper will be on the ethical aspects of the approaches.
This aligns with the framework of Responsible Research and Innovation (RRI), where innovation must strive for ``ethical acceptability, sustainability and societal desirability". \newline

The main goal of this paper is to initiate and exhaustive overview by focusing on a recent five-year period.
The paper is structured as follows: Section 2 dives into related works that provide overviews of the field.
Section 3 details the methodology used for information collection and the ethical audit assessment.
Section 4 presents the overview of situated visualization approaches along a timeline, analyzing shifts in terminology and categorization, followed by the ethical audit results.
Section 5 discusses the findings, implications for future research, and practical applications.
Finally, Section 6 concludes the paper by summarizing key insights and suggesting directions for future work.
