\section{Method}

The goal of this paper is to provide a systematic overview of situated visualization approaches, focusing on terminology, categorization, and ethical considerations.

\subsection{Defining ``Approach" and Scope}

In order to realize this overview, we first need to define what an ``approach'' is in this context.
A situated visualization approach is defined as a system that introduces a generalizable concept for visually representing data linked to its physical data referent.
An approach is generalizable if the contribution lies in the design itself, regardless of the specific data or location used.
This level of definition allows for a meaningful comparison, avoiding difficulties in analyzing non-generalizable examples.

To limit the scope and initiate what could become a truly exhaustive overview in the future, this paper focuses on a recent five-year period. The search was restricted to papers published between 2020 and 2025, aligning with the STAR nature of this paper.

\subsection{Information Collection}

This phase focuses on systematically gathering situated visualization approaches from academic literature.\\
\textbf{Methodology:} literature analysis

\subsubsection{Search Strategy and Keywords.}

The search strategy was designed to address the problem of "shifting sands" terminology by including a wide range of terms used interchangeably in the literature.
The literature search was conducted across two highly relevant digital libraries in the field: IEEE Explore and ACM Digital Library.

The search was conducted using the following keywords combined with the general terms `visualization':
\begin{itemize}
    \item Situated
    \item Embedded
    \item On-site
    \item In-situ
    \item Ambient
    \item Ubiquitous
    \item Contextual
\end{itemize}

The search was restricted to finding these keywords within the title and abstract of papers to ensure relevance.

\subsubsection{Selection Criteria.}

The systematic search resulted in a total of 60 papers. The following criteria were applied for inclusion:

\begin{itemize}
    \item The paper must present a generalizable situated visualization approach (as defined in Section 1.1)
    \item The approach must be explicitly linked to a physical data referent in the real world.
\end{itemize}

This filtering resulted in 34 relevant papers for further analysis.

\subsubsection{Exclusion Criteria.}

Papers were excluded primarily to ensure the set remained focused on the relationship between a visualization and its physical data referent. as per definition establish by Willet et al. (2017). The main reasons for exclusion were:

\begin{itemize}
    \item In-Situ Visualization for High-Performance Computing (HPC): Many papers using the term "In-Situ" were excluded because the term referred to computation happening on-site (close to the simulation), but the resulting visualization was still remote from the physical data referent.
    \item Non-generalizable: Paper that were primarily literature reviews, or surveys were excluded, as the goal was to identify and research presenting new approaches. These paper were however included in Section 2 (Related Work) for context.
\end{itemize}

\subsection{Ethical Audit Assessment}

Finally, an ethical audit was conducted on all selected approaches.
The assessment was performed using an Ethical Matrix Framework.
This framework requires reflecting on ethical concerns across specific stakeholder groups (the individual user, the environment/data referent, and society/community) using core ethical principles (Well-being/Beneficence, Autonomy/Freedom, and Justice/Fairness).

The purpose of this audit is to provide the user of this overview with more power to decide what approach fits their needs.
This focus is required because ethical implications and considerations are key in the framework of Responsible Research and Innovation (RRI), where innovation must strive for "ethical acceptability, sustainability and societal desirability".
The full ethical audit matrix is provided in the Appendix.
